\ssr{ВВЕДЕНИЕ}

Погода оказывает значительное влияние на общее эмоциональное состояние человека~\cite{windahl_real_time_2018}.
Это важно, например, для компьютерных игр, где множество факторов задействовано для создания нужного настроения в сценах, представленных игроку. Облака на небе -- один из наиболее значимых факторов, задающих настроение и формирующих пейзаж в компьютерных играх~\cite{windahl_real_time_2018}.

В итоге динамическая визуализация реалистичных облаков стала чрезвычайно востребованной для таких приложений, как компьютерные игры с открытым миром, системы имитации полетов и среды виртуальной реальности~\cite{sym10040125, windahl_real_time_2018}.

Цель работы -- разработка программного обеспечения для визуализации динамической ландшафтной сцены с облаками.

Для достижения поставленной цели необходимо решить следующие задачи:
\begin{itemize}
	\item проанализировать предметную область: рассмотреть известные подходы и алгоритмы для генерации и визуализации облаков и ландшафта;
	\item описать оптическую модель облаков;
	\item проанализировать и выбрать алгоритмы решения основных задач компьютерной графики: удаления невидимых линий и поверхностей, учёта теней и освещения;
	\item спроектировать программное обеспечение, позволяющее визуализировать динамическую ландшафтную сцену с облаками;
	\item выбрать средства реализации этого программного обеспечения и создать его;
	\item провести исследование разработанного программного обеспечения.
\end{itemize}

