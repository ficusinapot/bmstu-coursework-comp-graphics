\ssr{ЗАКЛЮЧЕНИЕ}

В ходе работы были решены задачи, связанные с визуализацией ландшафтной сцены с облаками. Для этого был выбран алгоритм обратной трассировки лучей для визуализации объемных облаков, а также метод аппроксимации ландшафта примитивами. Для процедурной генерации ландшафта применён шум Перлина, а для облаков — шум Ворли--–Вороного.

Была предложена оптическая модель облаков, а также выбраны алгоритмы для решения ключевых задач компьютерной графики. Для устранения невидимых линий и поверхностей использован Z-буфер, а для обработки теней и освещения — закон Ламберта и метод Фонга.

Разработано программное обеспечение, произведена его декомпозиция с описанием алгоритмов. Было выполнено модульное тестирование, и программа успешно прошла все тесты. Покрытие программы тестами составляет 19.83\%.

Также были проанализированы результаты исследования, в ходе которого проведена оценка производительности алгоритма визуализации облаков. Оценка респондентами качества реализации с количеством точек на трассируемом луче равным 240 составила 7.8/10. Однако время выполнения алгоритма (265 мс на кадр) требует оптимизаций для достижения производительности в 30 кадров в секунду.

При дальнейшем развитии работы потребуется улучшение алгоритма визуализации облака с целью повышения производительности, например, за счёт распараллеливания на графической карте. Усложнение ландшафта и улучшение качества облаков, например, за счёт использования других оптических моделей~\cite{guerrilla_volumetric_cloudscapes_2023} и других шумов, также могут стать направлениями дальнейшего развития работы.