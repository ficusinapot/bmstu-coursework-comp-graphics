\chapter{Аналитическая часть}
В аналитической части будут формализованы задачи и объекты сцены, определены геометрические и оптические характеристики объектов сцены. Также будут проанализированы и описаны алгоритмы, используемые для визуализации ландшафтной сцены с облаками. Будут установлены допустимые диапазоны и ограничения, накладываемые на входные данные.

\section{Формализация задачи и объектов}

Объектами сцены являются:
\begin{enumerate}
	\item \textbf{Облака (облачный пейзаж)}
	\begin{itemize}
		\item Высота, на которой находятся облака;
		\item Скорость движения облаков по горизонту;
		\item Кучность: степень сжатия и плотности облаков, что влияет на их внешний вид и отбрасываемую тень.
		\item Плотность: определяет, сколько солнечного света облака могут заблокировать, что влияет на освещение ландшафта.
	\end{itemize}
	\item \textbf{Ландшафт (ландшафтный пейзаж)} -- 
	\begin{itemize}
		\item Рельеф: плоский равнинный.
		\item Материалы и текстуры: характеристики поверхности, такие как цвет и отражательная способность.
		\item Освещение от солнца и теней: ландшафт получает освещение, которое зависит от плотности облаков и положения солнца, а также отбрасываемых теней.
	\end{itemize}
	\item \textbf{Бесконечно удаленный источник света (солнце)} --
	\begin{itemize}
		\item Расположение: определяется положением на небесной сфере. Положение солнца влияет на длину и направление теней.
		\item Интенсивность: определяет, насколько ярко освещен ландшафт, также зависит от плотности облаков.
	\end{itemize}
	\item \textbf{Наблюдатель (камера)} --
	\begin{itemize}
		\item Расположение: координаты и угол обзора камеры, позволяющие наблюдать сцену с разных ракурсов.
		\item Поле зрения: угол обзора, влияющий на широту сцены.
	\end{itemize}
\end{enumerate}

Определение диапазонов и ограничений:
\begin{itemize}
	\item \textbf{Высота облаков:} от 1000 до 3000 метров.
	\item \textbf{Скорость облаков:} от 0 до 50 км/ч.
	\item \textbf{Плотность облаков:} значение от 0 (полностью прозрачные) до 1 (непрозрачные).
	\item \textbf{Положение солнца:} угол наклона от $0^{\circ}$ до $90^{\circ}$ над горизонтом и азимутальный угол от $0^{\circ}$ до $180^{\circ}$.
	\item \textbf{Пространственное перемещение} осущствляется только для таких объектов, как камера и солнце.
\end{itemize}

\section{Алгоритмы генерации облаков}
Существует несколько подходов к реализации облаков \cite{unigine_volumetric_clouds_2022}:	\begin{itemize}
\item \textbf{Геометрический:} облака представляют собой, например, набор треугольников, сфер или прямоугольников. Геометрический подход к созданию облаков имеет смысл в оперделенной стилистике изображения.
\item \textbf{Двумерная текстура:} простой и малозатратный подход, но такая статичная картинка имеет смысл только как дальнеплановые статичные изображения, через которые, например, нельзя пролететь сквозь. К тому же такие облака не могут производить тени.
\item \textbf{Объемные} \textit{(volumetric)}: динамические облака, с которыми можно взаимодействовать и которые способны производить тени. Именно поэтому такие облака будут реализованы в данной работе.
\end{itemize}

Заключим требования к алгоритму:
\begin{itemize}
	\item Облака должны быть объемные;
	\item Облака должны генерироваться процедурно;
	\item Должен быть быстродействующим.
\end{itemize}
\subsection{Жидкостная симуляция} 

Использование жидкостной симуляции для создания объемных облаков: создать простые объекты (сферы, шары), вокселизировать их и рассматривать их как жидкость, получая похожие на объемные облака фигуру \cite{guerrilla_volumetric_cloudscapes_2023}.

Недостатки: 
\begin{itemize}
	\item Алгоритм медленный;
	\item Сложность контроля генерации;
	\item Сложность реализации.
\end{itemize}
\subsection{Повоксельная генерация}
Алгоритм заключается в генерации ограничивающего параллелепипеда (bounding box), состоящего из вокселей, хранящих информацию о цвете. \cite{guerrilla_volumetric_cloudscapes_2023}.
Преимущества:
\begin{itemize} 
	\item Хорошо сочетается с алгоритмом построением теней
\end{itemize}
Недостатки: 
\begin{itemize} 
	\item Высокие затраты памяти; 
	\item Сложность обработки большого количества вокселей в реальном времени; 
	\item Необходимость оптимизаций для обработки больших объемов. 
\end{itemize}
\subsection{Генерация на основе обратной трассировки лучей}
Из точки наблюдателя для каждого пикселя грани высчитывается его итоговый цвет \cite{Patapom2013}.
Алгоритм также опирается на ограничивающий параллелепипед, но вместо этого визуализируются лишь видимые грани параллелипипеда. Вместо вычисления каждого вокселя, алгоритм ориентируется на пиксели, видимые пользователю, и рассчитывает итоговые цвета только для них. 
\begin{itemize} 
	\item Также хорошо сочетается с алгоритмом построением теней;
	\item Меньшие затраты памяти; 
	\item Сниженные вычислительные затраты благодаря обработке только видимых пикселей.
\end{itemize}
Обратная трассировка лучей показывает преимущество перед повоксельной и жидкостной генерациях, так как обрабатывает только видимые пиксели, что снижает вычислительные затраты и экономит память, что необходимо при формировании динамического изображения.

\section{Алгоритм генерации ландшафта}
 Одним из основных методов генерации ландшафта является использование шумов и аппроксимация примитивами.

\subsection{Использование шумов}

Шумы служат основой для создания естественных и органически выглядящих ландшафтов. Эти алгоритмы генерируют псевдослучайные значения, которые могут быть использованы для создания разнообразных элементов ландшафта, таких как высота, текстуры и цвет.

Один из наиболее распространенных способов использования шумов в генерации ландшафта заключается в создании высотной карты. Высотная карта — это двумерный массив значений, где каждое значение соответствует высоте точки на поверхности. 
Используя шумы с низкой амплитудой и частотой можно получить \textit{равнинный ландшафт}, который и необходимо реализовать по техническому заданию.

\subsection{Аппроксимация примитивами}

Для представления сгенерированного ландшафта используются примитивы, и наиболее распространенным вариантом являются треугольники.

Основные шаги в использовании треугольников для аппроксимации ландшафта включают:
\begin{itemize}
	\item Создание сетки: формирование сетки, состоящей из вершин, соединенных ребрами. Каждая вершина соответствует точке в высотной карте, а ребра образуют треугольники.
	\item Обработка вершин: применение значений высоты из высотной карты к вершинам сетки для создания рельефа.
\end{itemize}

\section{Модели освещения}
\subsection{Закон Бугера~---~Ламберта~---~Бера}
Для облаков некоторая часть света рассеивается от направления распространения, а еще большее количество поглощается каплями воды и молекулами озона, но остается часть, которая продолжает движение без изменений.

\textit{Закон Бугера---Ламберта---Бера} определяет ослабление пучка света при поглощении средой.
\begin{equation}
	\label{eq:beers-law}
	I_l=I_{o}e^{-k_{\lambda }l},
\end{equation} где 
{$I_{0}$} — интенсивность света на входе в вещество, 
$k_\lambda$ — показатель поглощения.
\subsection{Закон Ламберта}
Матовые поверхности обладают свойством диффузного отражения, т. е.
равномерного по всем направлениям рассеивания света, благодаря чему поверхности визуально имеют одинаковую яркость независимо от угла обзора \cite{rodgers1989algorithms}

\textit{Закон Ламберта} определяет интенсивность диффузного отражения света:
\begin{equation}
	\label{eq:lambert-law}
	I_d=I_{p}K_d\cos(\theta),
\end{equation} где 
{$I_{0}$} — интенсивность света на входе в вещество, 
$\theta$ — угол между направлением точечного источника светаа
интенсивности $I_p$ и нормалью $\vec{N}$ к поверхности.

\section{Алгоритм построения теней облаков}
Тени от облаков зависят только от положения на поверхности, что делает их независимыми от точки зрения \cite{rodgers1989algorithms}. Для объемных облаков при построении их теней аналогично используется обратная трассировка лучей: при движении луча от поверхности к облакам определяется суммарная
плотность облаков по пути, чтобы вычислить, сколько света блокируется \cite{shadows2023}.

Таким образом, учитывая закон Бугера—Ламберта—Бера и закон Ламберта, можем записать итоговую формулу для расчета интенсивности света на поверхности:
\begin{equation}
	I_s = I_0 K_d \cos(\theta) e^{-k_{\lambda} l}.
\end{equation}
где
	 $I_s$ — итоговая интенсивность света на поверхности,
	 $I_0$ — интенсивность света от источника,
	 $K_d$ — коэффициент диффузного отражения,
	 $\theta$ — угол между направлением света и нормалью к поверхности,
	 $k_{\lambda}$ — показатель поглощения,
	 $l$ — расстояние, пройденное светом через облака.

\section*{Вывод}
В аналитической части формализованы задачи и объекты сцены, определены геометрические и оптические характеристики объектов сцены. Также проанализированы и описаны алгоритмы, используемые для визуализации ландшафтной сцены с облаками. Установлены допустимые диапазоны и ограничения, накладываемые на входные данные.
Был выбран алгоритм использующий обратную трассировку лучей для генерации объемных облаков, а также алгоритм для построения ландшафта с помощью аппроксимацией примитивами.
